\documentclass{beamer}
\mode<presentation>
\usepackage{amsmath}
\usepackage{amssymb}
%\usepackage{advdate}
\usepackage{adjustbox}
\usepackage{subcaption}
\usepackage{enumitem}
\usepackage{multicol}
\usepackage{mathtools}
\usepackage{listings}
\usepackage{url}
\def\UrlBreaks{\do\/\do-}
\usetheme{metropolis}
\setbeamertemplate{footline}
{
  \leavevmode%
  \hbox{%
  \begin{beamercolorbox}[wd=\paperwidth,ht=2.25ex,dp=1ex,right]{author in head/foot}%
    \insertframenumber{} / \inserttotalframenumber\hspace*{2ex} 
  \end{beamercolorbox}}%
  \vskip0pt%
}
\setbeamertemplate{navigation symbols}{}

\providecommand{\nCr}[2]{\,^{#1}C_{#2}} % nCr
\providecommand{\nPr}[2]{\,^{#1}P_{#2}} % nPr
\providecommand{\mbf}{\mathbf}
\providecommand{\pr}[1]{\ensuremath{\Pr\left(#1\right)}}
\providecommand{\qfunc}[1]{\ensuremath{Q\left(#1\right)}}
\providecommand{\sbrak}[1]{\ensuremath{{}\left[#1\right]}}
\providecommand{\lsbrak}[1]{\ensuremath{{}\left[#1\right.}}
\providecommand{\rsbrak}[1]{\ensuremath{{}\left.#1\right]}}
\providecommand{\brak}[1]{\ensuremath{\left(#1\right)}}
\providecommand{\lbrak}[1]{\ensuremath{\left(#1\right.}}
\providecommand{\rbrak}[1]{\ensuremath{\left.#1\right)}}
\providecommand{\cbrak}[1]{\ensuremath{\left\{#1\right\}}}
\providecommand{\lcbrak}[1]{\ensuremath{\left\{#1\right.}}
\providecommand{\rcbrak}[1]{\ensuremath{\left.#1\right\}}}
\theoremstyle{remark}
\newtheorem{rem}{Remark}
\newcommand{\sgn}{\mathop{\mathrm{sgn}}}
\providecommand{\abs}[1]{\left\vert#1\right\vert}
\providecommand{\res}[1]{\Res\displaylimits_{#1}} 
\providecommand{\norm}[1]{\lVert#1\rVert}
\providecommand{\mtx}[1]{\mathbf{#1}}
\providecommand{\mean}[1]{E\left[ #1 \right]}
\providecommand{\fourier}{\overset{\mathcal{F}}{ \rightleftharpoons}}
%\providecommand{\hilbert}{\overset{\mathcal{H}}{ \rightleftharpoons}}
\providecommand{\system}{\overset{\mathcal{H}}{ \longleftrightarrow}}
	%\newcommand{\solution}[2]{\textbf{Solution:}{#1}}
%\newcommand{\solution}{\noindent \textbf{Solution: }}
\providecommand{\dec}[2]{\ensuremath{\overset{#1}{\underset{#2}{\gtrless}}}}
\newcommand{\myvec}[1]{\ensuremath{\begin{pmatrix}#1\end{pmatrix}}}
\let\vec\mathbf

\lstset{
%language=C,
frame=single, 
breaklines=true,
columns=fullflexible
}

\numberwithin{equation}{section}

\title{Numerical Solutions to Equations}
\author{Aditya Tripathy\\ Dept. of Electrical Engg.,\\IIT Hyderabad.}

\date{\today} 
\begin{document}

\begin{frame}
\titlepage
\end{frame}

\section*{Outline}
\begin{frame}
\frametitle{Table Of Contents}
\tableofcontents
\end{frame}
\section{Problem}
\begin{frame}
\frametitle{Problem Statement}
Solve the equation 
$x^2 - 2x = -2(3 - x)$ or $x^2 - 4x + 6 = 0$
%A circle $C$ passes through 
%\begin{equation} 
%\vec{P}=\myvec{-2\\ 4} 
%\label{eq:circle_7_p}
%\end{equation} 
%and touches the $y$-axis at 
%\begin{equation} 
%\vec{Q}=\myvec{0\\ 2}. 
%\label{eq:circle_7_q}
%\end{equation}
%Which one of the  following equations can represent a diameter of this circle?
%\begin{enumerate}[label=(\roman*)]
%\begin{multicols}{2}
%\setlength\itemsep{1em}
%\item $\myvec{4 & 5}\vec{x} = 6 $
%\item $\myvec{2 & -3}\vec{x} +10 = 0 $
%\item $\myvec{3 & 4}\vec{x} = 3 $
%\item $\myvec{5 & 2}\vec{x} +4= 0 $
%\end{multicols}
%\end{enumerate}
\end{frame}

%\subsection{Literature}
\section{Characteristic Polynomial}
\begin{frame}
  \frametitle{Characteristic Polynomial}
  Recall that for a matrix $A$ of order $n$, the characteristic equation is given by,
\begin{align}
  det\brak{A - \lambda I} = a_n \lambda ^n + a_{n-1} \lambda ^{n-1} \cdots + a_0 = 0
 \end{align}
 Recall that the solutions to the characteristic polynomial are the eigenvalues of the the matrix
 A. So if we can somehow construct the corresponding matrix from the characteristic polynomial,
 our job will be finished, since we can find the eigen values using the QR alogorithm.

%\framesubtitle{Literature}
%Let $\vec{O}$ be the centre of $C$. Then the equation of the normal, OQ is
%\begin{align}
%%\vec{x}^T\vec{x}-2\vec{O}^T\vec{x} +F = 0
%\myvec{0 & 1}\brak{\vec{O}-\vec{Q}} &= 0
%\nonumber \\ 
%\implies \myvec{0 & 1}\vec{O} = 2
%\label{eq:circle_7_o1}
%\end{align}
%%
%Also, 
%%Substituting \eqref{eq:circle_7_p} in \eqref{eq:circle_7_c}, 
%\begin{align}
%\norm{\vec{O}-\vec{P}}^2&=\norm{\vec{O}-\vec{Q}}^2 
%\nonumber \\
%\implies 2\brak{\vec{P}-\vec{Q}}^T\vec{O} &= \norm{\vec{P}}^2-\norm{\vec{Q}}^2 
%\nonumber \\
%\text{or, } \myvec{1 & -1}\vec{O} &= -4
%\label{eq:circle_7_o2}
%\end{align}
%%
%\eqref{eq:circle_7_o1} and \eqref{eq:circle_7_o2} result in the matrix equation
%\begin{align}
%\myvec{1 & -1 \\ 0 & 1}\vec{O} = \myvec{-4\\2}
%\label{eq:circle_7_matrix}
%\end{align}
%yielding the augmented matrix
%\begin{align}
%\myvec{1 & -1 & -4\\ 0 & 1 & 2} \leftrightarrow \myvec{1 & 0 & -2\\ 0 & 1 & 2}\implies \vec{O} = \myvec{-2 \\2}
%\label{eq:circle_7_o}
%\end{align}
%%
\end{frame}
\begin{frame}
\frametitle{Companion Matrix}
A matrix is said to be the companion of a polynomial $f\brak{x}$ if 
$det\brak{A - \lambda I} = 0 \implies f\brak{x} = 0$. 
\newline
For,
\begin{align}
  f\brak{x} = c_0 + c_1 x \cdots + x^n
\end{align}
The companion matrix is,
\begin{align}
  \myvec{
    0 & 0 & \cdots & 0 & -c_0\\
    1 & 0 & \cdots & 0 & -c_1\\
    0 & 1 & \cdots & 0 & -c_2\\
    \vdots & \vdots & \ddots & \vdots & \vdots\\
    0 & 0 & \cdots & 1 & -c_{n-1}\\
  }
\end{align}
%\begin{enumerate}[label=(\roman*)]
%\item $\myvec{4 & 5}\vec{O} = 2 \ne 6 $. Incorrect.
%\vfill
%\item $\myvec{2 & -3}\vec{O} +10 = 0 $. Correct.
%\vfill
%\item $\myvec{3 & 4}\vec{O} = 2 \ne 3 $.  Incorrect.
%\vfill
%\item $\myvec{5 & 2}\vec{O} +4= -2 \ne 0 $. Incorrect
%\end{enumerate}

\end{frame}
\begin{frame}
\frametitle{Our Companion Matrix}
\begin{align}
  \myvec{
    0 & -6\\
    1 & 4
  }
\end{align}
Using the QR algorithm we can now solve for the eigenvalues and thus the solutions 
for the given equation.
\end{frame}
\section{Understanding QR Algorithm}
\begin{frame}
\frametitle{Similarity Transform}
Two matrices $A, B \in \mathbb{C}^{n \times n}$ are said to be similar if there exists an invertible matrix $P \in \mathbb{C}^{n \times n}$ such that
    \begin{align}
        B = P^{-1}AP
    \end{align}
\end{frame}
\begin{frame}
\frametitle{Main Idea}
The idea is to apply similarity transforms strategically to convert given matrix to Upper 
triangular matrix, because diagonal entries are the eigenvalues of the original matrix.
\end{frame}
\begin{frame}
  \frametitle{Intermediate Steps}
The steps to solve QR quickly :
\begin{align}
  \text{1. Convert original matrix to an intermediate form (hessenberg form)}\\
  \text{2. Apply givens rotation}\\
  \text{3. Diagonal elements are eigenvalues}\\
\end{align}
\end{frame}
\section{Conversion To Hessenberg Form}
\begin{frame}
\frametitle{Householder Reflections}
A matrix of the form
\begin{align}
    P = I - 2\textbf{uu}^*\\
\end{align}
is called a Householder reflector.
A task that we repeatedly want to carry out with Householder reflectors is to transform
a vector \textbf{x} on a multiple of \textbf{$e_1$}
\begin{align}
    P\textbf{x} = \textbf{x} - \textbf{u}(2\textbf{u}^*\textbf{x}) = \alpha \textbf{$e_1$}\\
\end{align}
Since P is unitary, we must have $\alpha = \rho 
||\textbf{x}||$, where $\rho \in \mathbb{C}$ has absolute value one. Therefore,
\begin{align}
    \textbf{u} = \frac{\textbf{x} - \rho||\textbf{x}||\textbf{$e_1$}}{||\textbf{x} - \rho||\textbf{x}||\textbf{$e_1$}||}
\end{align}
\end{frame}
\begin{frame}
\frametitle{Householder Reflections}
We can freely choose $\rho$ provided that $|\rho| = 1$. Let $x_1 = |x_1|e^{i\textbf{$u_k$}^*\phi}$.To avoid numerical cancellation we set $\rho = -e^{i\phi}$
\end{frame}
\begin{frame}
\frametitle{Visualizing The process}
\begin{align}  
    P_1 = \begin{bmatrix}
        1 & 0 & 0 & 0 & 0 \\
        0 & \times & \times& \times& \times\\
        0 & \times & \times& \times& \times\\
        0 & \times & \times& \times& \times\\
        0 & \times & \times& \times& \times\\
    \end{bmatrix} = \begin{bmatrix}
        1 & \textbf{0}^{\top}\\
        \textbf{0} & I_4 - 2u_1u_1^*
    \end{bmatrix}    
\end{align}
\begin{multline}
    P_1^*AP_1 = \begin{bmatrix}
        \times & \times & \times& \times& \times\\
        \times & \times & \times& \times& \times\\
        0 & \times & \times& \times& \times\\
        0 & \times & \times& \times& \times\\
        0 & \times & \times& \times& \times\\
    \end{bmatrix} \xrightarrow[P_2*/*P_2]{} \begin{bmatrix}
        \times & \times & \times& \times& \times\\
        \times & \times & \times& \times& \times\\
        0 & \times & \times& \times& \times\\
        0 & 0 & \times& \times& \times\\
        0 & 0 & \times& \times& \times\\
    \end{bmatrix} 
    \end{multline}

\end{frame}
\begin{frame}
\frametitle{Returning to problem}
Since a 2x2 matrix is always upper hessenber, we need not do the above steps
\end{frame}
\begin{frame}
\frametitle{Step 2: Apply Givens Rotation}
Givens rotations are used to zero specific elements of a vector or matrix by rotating the vector in the plane of two coordinates. A Givens rotation matrix is defined as:
\[
G(i, j, \theta) =
\begin{bmatrix}
1 & \cdots & 0 & \cdots & 0 & \cdots & 0 \\
\vdots & \ddots & \vdots & \ddots & \vdots & \ddots & \vdots \\
0 & \cdots & c & \cdots & s & \cdots & 0 \\
\vdots & \ddots & \vdots & \ddots & \vdots & \ddots & \vdots \\
0 & \cdots & -\overline{s} & \cdots & \overline{c} & \cdots & 0 \\
\vdots & \ddots & \vdots & \ddots & \vdots & \ddots & \vdots \\
0 & \cdots & 0 & \cdots & 0 & \cdots & 1
\end{bmatrix},
\]
where 
\begin{align}
    c_k = \frac{\overline{x_i}}{\sqrt{|x_i|^2 + |x_j|^2}},
    s_k = \frac{\overline{x_j}}{\sqrt{|x_i|^2 + |x_j|^2}}   
\end{align}
\end{frame}
\begin{frame}
\frametitle{Step 2: Apply Givens Rotation}
\begin{align}
  c_k &= \frac{A_{11}}{\sqrt{A_{11}^2 + A_{12}^2}} = \frac{0}{\sqrt{0^2 + 1^2}} = 0\\
  s_k &= \frac{A_{12}}{\sqrt{A_{11}^2 + A_{12}^2}}= \frac{0}{\sqrt{0^2 + 1^2}} = 1\\
  G &=\myvec{0 & 1\\-1 & 0}\\
  G^{\top}AG &= \myvec{0 & -1\\1 & 0}\myvec{0 & -6\\1 & 4}\myvec{0 & 1\\-1 & 0}\\
  &= \myvec{4 & -1\\6 & 0}
\end{align}
\end{frame}
\begin{frame}
  \frametitle{Failure?}
The QR algorithm has failed to converge to an upper triangular matrix, which is to be expected.
A matrix with real entries but complex eigenvalues, cannot be transformed by any sort of matrix 
multipliation of conjugation.
\end{frame}
\begin{frame}
  \frametitle{Jordan Blocks}
  \begin{align}
    \begin{bmatrix}
        a_0 & & & &  \\
         & a_1 & & &  \\
         & & a_2 & & \\
         & &  & a_3 &a_4\\
         & & & a_5& a_6\\
    \end{bmatrix}
\end{align}
Sometimes the QR algorithm ends like this. The propper diagonal elements are the true eigenvalues, and the eigenvalues of the 2$\times$2 block is the complex conjugate pair of eigenvalues
\end{frame}
\begin{frame}
  \frametitle{Code Output}
  \begin{align}
  \text{Upper Hessenberg Matrix}\\
  \text{0.000000000 +0.000000000i -6.000000000 +0.000000000i }\\
  \text{1.000000000 +0.000000000i 4.000000000 +0.000000000i }\\
  \text{eigenvalue 1: 2.000000 + 1.414214i}\\
  \text{eigenvalue 2: 2.000000 + -1.414214i}\\
  \end{align}
 \end{frame}
%\section{Plot}
\end{document}
