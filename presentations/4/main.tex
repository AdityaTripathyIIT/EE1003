\documentclass{beamer}
\mode<presentation>
\usepackage{amsmath}
\usepackage{amssymb}
%\usepackage{advdate}
\usepackage{adjustbox}
\usepackage{subcaption}
\usepackage{enumitem}
\usepackage{multicol}
\usepackage{mathtools}
\usepackage{listings}
\usepackage{url}
\def\UrlBreaks{\do\/\do-}
\usetheme{metropolis}
\setbeamertemplate{footline}
{
  \leavevmode%
  \hbox{%
  \begin{beamercolorbox}[wd=\paperwidth,ht=2.25ex,dp=1ex,right]{author in head/foot}%
    \insertframenumber{} / \inserttotalframenumber\hspace*{2ex} 
  \end{beamercolorbox}}%
  \vskip0pt%
}
\setbeamertemplate{navigation symbols}{}

\providecommand{\nCr}[2]{\,^{#1}C_{#2}} % nCr
\providecommand{\nPr}[2]{\,^{#1}P_{#2}} % nPr
\providecommand{\mbf}{\mathbf}
\providecommand{\pr}[1]{\ensuremath{\Pr\left(#1\right)}}
\providecommand{\qfunc}[1]{\ensuremath{Q\left(#1\right)}}
\providecommand{\sbrak}[1]{\ensuremath{{}\left[#1\right]}}
\providecommand{\lsbrak}[1]{\ensuremath{{}\left[#1\right.}}
\providecommand{\rsbrak}[1]{\ensuremath{{}\left.#1\right]}}
\providecommand{\brak}[1]{\ensuremath{\left(#1\right)}}
\providecommand{\lbrak}[1]{\ensuremath{\left(#1\right.}}
\providecommand{\rbrak}[1]{\ensuremath{\left.#1\right)}}
\providecommand{\cbrak}[1]{\ensuremath{\left\{#1\right\}}}
\providecommand{\lcbrak}[1]{\ensuremath{\left\{#1\right.}}
\providecommand{\rcbrak}[1]{\ensuremath{\left.#1\right\}}}
\theoremstyle{remark}
\newtheorem{rem}{Remark}
\newcommand{\sgn}{\mathop{\mathrm{sgn}}}
\providecommand{\abs}[1]{\left\vert#1\right\vert}
\providecommand{\res}[1]{\Res\displaylimits_{#1}} 
\providecommand{\norm}[1]{\lVert#1\rVert}
\providecommand{\mtx}[1]{\mathbf{#1}}
\providecommand{\mean}[1]{E\left[ #1 \right]}
\providecommand{\fourier}{\overset{\mathcal{F}}{ \rightleftharpoons}}
%\providecommand{\hilbert}{\overset{\mathcal{H}}{ \rightleftharpoons}}
\providecommand{\system}{\overset{\mathcal{H}}{ \longleftrightarrow}}
	%\newcommand{\solution}[2]{\textbf{Solution:}{#1}}
%\newcommand{\solution}{\noindent \textbf{Solution: }}
\providecommand{\dec}[2]{\ensuremath{\overset{#1}{\underset{#2}{\gtrless}}}}
\newcommand{\myvec}[1]{\ensuremath{\begin{pmatrix}#1\end{pmatrix}}}
\let\vec\mathbf

\lstset{
%language=C,
frame=single, 
breaklines=true,
columns=fullflexible
}

\numberwithin{equation}{section}

\title{Numerical Solutions to Equations}
\author{Aditya Tripathy\\ Dept. of Electrical Engg.,\\IIT Hyderabad.}

\date{\today} 
\begin{document}

\begin{frame}
\titlepage
\end{frame}

\section*{Outline}
\begin{frame}
\frametitle{Table Of Contents}
\tableofcontents
\end{frame}
\section{Problem}
\begin{frame}
\frametitle{Problem Statement}
Solve the equation 
$x^2 - 2x = -2(3 - x)$ or $x^2 - 4x + 6 = 0$
%A circle $C$ passes through 
%\begin{equation} 
%\vec{P}=\myvec{-2\\ 4} 
%\label{eq:circle_7_p}
%\end{equation} 
%and touches the $y$-axis at 
%\begin{equation} 
%\vec{Q}=\myvec{0\\ 2}. 
%\label{eq:circle_7_q}
%\end{equation}
%Which one of the  following equations can represent a diameter of this circle?
%\begin{enumerate}[label=(\roman*)]
%\begin{multicols}{2}
%\setlength\itemsep{1em}
%\item $\myvec{4 & 5}\vec{x} = 6 $
%\item $\myvec{2 & -3}\vec{x} +10 = 0 $
%\item $\myvec{3 & 4}\vec{x} = 3 $
%\item $\myvec{5 & 2}\vec{x} +4= 0 $
%\end{multicols}
%\end{enumerate}
\end{frame}

%\subsection{Literature}
\section{Fixed Point Iteration}
\subsection{}
\begin{frame}
  \frametitle{Towards the Update Eqn}
  Take an initial guess $x_0$. The update difference equation will use the following function:
  \begin{align}
    x = g\brak{x}
  \end{align}
  For our problem,
  \begin{align}
    g\brak{x} = \frac{1}{4}x^2 + \frac{3}{2}
  \end{align}
%\framesubtitle{Literature}
%Let $\vec{O}$ be the centre of $C$. Then the equation of the normal, OQ is
%\begin{align}
%%\vec{x}^T\vec{x}-2\vec{O}^T\vec{x} +F = 0
%\myvec{0 & 1}\brak{\vec{O}-\vec{Q}} &= 0
%\nonumber \\ 
%\implies \myvec{0 & 1}\vec{O} = 2
%\label{eq:circle_7_o1}
%\end{align}
%%
%Also, 
%%Substituting \eqref{eq:circle_7_p} in \eqref{eq:circle_7_c}, 
%\begin{align}
%\norm{\vec{O}-\vec{P}}^2&=\norm{\vec{O}-\vec{Q}}^2 
%\nonumber \\
%\implies 2\brak{\vec{P}-\vec{Q}}^T\vec{O} &= \norm{\vec{P}}^2-\norm{\vec{Q}}^2 
%\nonumber \\
%\text{or, } \myvec{1 & -1}\vec{O} &= -4
%\label{eq:circle_7_o2}
%\end{align}
%%
%\eqref{eq:circle_7_o1} and \eqref{eq:circle_7_o2} result in the matrix equation
%\begin{align}
%\myvec{1 & -1 \\ 0 & 1}\vec{O} = \myvec{-4\\2}
%\label{eq:circle_7_matrix}
%\end{align}
%yielding the augmented matrix
%\begin{align}
%\myvec{1 & -1 & -4\\ 0 & 1 & 2} \leftrightarrow \myvec{1 & 0 & -2\\ 0 & 1 & 2}\implies \vec{O} = \myvec{-2 \\2}
%\label{eq:circle_7_o}
%\end{align}
%%
\end{frame}
\begin{frame}
\frametitle{Update Equation}
Now the update equation will be,
\begin{align}
    x_{n+1} = g\brak{x_n} 
\end{align}
When we try to run the iterations however, we realize that whatever be the initial guess, 
the subsequent updated values grow without bound. This is becaue of the following theorem
%\begin{enumerate}[label=(\roman*)]
%\item $\myvec{4 & 5}\vec{O} = 2 \ne 6 $. Incorrect.
%\vfill
%\item $\myvec{2 & -3}\vec{O} +10 = 0 $. Correct.
%\vfill
%\item $\myvec{3 & 4}\vec{O} = 2 \ne 3 $.  Incorrect.
%\vfill
%\item $\myvec{5 & 2}\vec{O} +4= -2 \ne 0 $. Incorrect
%\end{enumerate}

\end{frame}
\begin{frame}
\frametitle{Theorem}
    Let $x = s$ be a solution of $x = g\brak{x}$ and suppose that $g$ has a continuous
    derivative in some interval $J$ containing $s$. Then if $\abs{g^{\prime}} \le K < 1$ in $J$,
    the iteration process defined  above converges for any $x_0$ in $J$. The limit of the sequence
    $\sbrak{x_n}$ is s\\
\end{frame}
\begin{frame}
\frametitle{Conclusion 1}
Since there is no solution (evident by quadratic formula) there exists no interval J for which
the process converges to a point.\\
\end{frame}
\section{Newton Raphson Method}
\begin{frame}
\frametitle{The Method}
tart with an initial guess $x_0$, and then run the following logical loop,
\begin{align}
    x_{n+1} = x_n - \frac{f\brak{x_n}}{f^{\prime}\brak{x_n}} 
\end{align}
where ,
\begin{align}
    f\brak{x} = x^2 - 4x + 6\\
    f^{\prime}\brak{x} = 2x-4
\end{align}
\end{frame}
\begin{frame}
\frametitle{Behaviour of the method}
The behaviour shown here is that regardless of which guess we take, it reaches a point of 
extrema(derivative $\approx$ 0) and then the process halts, or the updated point grow with bound.\\
\end{frame}

\section{The Solution}
\begin{frame}
\frametitle{Complex initital guesses!}
To get the complex solutions, however , we can just take the initial guess point to be a 
random complex number.
\end{frame}
\begin{frame}
\frametitle{Code Output}
\begin{align}
  \textnormal{Running Newton iterations:}\\
  \textnormal{x got too big}\\
  \textnormal{Trying fixed point iterations:}\\
  \textnormal{x got too big}\\
  \textnormal{Trying complex Newton's iterations:}\\
  \textnormal{Solution = 2.000000 + 1.414214 i}\\
\end{align}
\end{frame}
\begin{frame}
  \frametitle{Code Output}

  And on a second run, 
  \begin{align}
    \textnormal{Running Newton iterations:}\\
    \textnormal{Failure}\\
    \textnormal{Trying fixed point iterations:}\\
    \textnormal{x got too big}\\
    \textnormal{Trying complex Newton's iterations:}\\
    \textnormal{Solution = 2.000000 + -1.414214 i}\\
  \end{align}
\end{frame}
%\section{Plot}
\end{document}
