\documentclass[journal]{IEEEtran}
\usepackage[a5paper, margin=10mm, onecolumn]{geometry}
%\usepackage{lmodern} % Ensure lmodern is loaded for pdflatex
\usepackage{tfrupee} % Include tfrupee package

\setlength{\headheight}{1cm} % Set the height of the header box
\setlength{\headsep}{0mm}     % Set the distance between the header box and the top of the text

\usepackage{gvv-book}
\usepackage{gvv}
\usepackage{cite}
\usepackage{amsmath,amssymb,amsfonts,amsthm}
\usepackage{algorithmic}
\usepackage{graphicx}
\usepackage{textcomp}
\usepackage{xcolor}
\usepackage{txfonts}
\usepackage{listings}
\usepackage{enumitem}
\usepackage{mathtools}
\usepackage{gensymb}
\usepackage{comment}
\usepackage[breaklinks=true]{hyperref}
\usepackage{tkz-euclide} 
\usepackage{listings}
% \usepackage{gvv}                                        
\def\inputGnumericTable{}                                 
\usepackage[latin1]{inputenc}                                
\usepackage{color}                                            
\usepackage{array}                                            
\usepackage{longtable}                                       
\usepackage{calc}                                             
\usepackage{multirow}                                         
\usepackage{hhline}                                           
\usepackage{ifthen}                                           
\usepackage{lscape}
\begin{document}

\bibliographystyle{IEEEtran}
\vspace{3cm}

\title{10.3.1.1}
\author{EE24BTECH11001 - Aditya Tripathy}
 \maketitle
% \newpage
% \bigskip
{\let\newpage\relax\maketitle}

\renewcommand{\thefigure}{\theenumi}
\renewcommand{\thetable}{\theenumi}
\setlength{\intextsep}{10pt} % Space between text and floats


\numberwithin{equation}{enumi}
\numberwithin{figure}{enumi}
\renewcommand{\thetable}{\theenumi}


\textbf{Question}:\\
Find the roots of the equation $2x^2 - 3x + 5 = 0$
\\
\textbf{Solution: }\\
Below are two methods to find the solutions of this quadratic equation,

Fixed Point Iterations:
Take an initial guess $x_0$. The update difference equation will use the following function:
\begin{align}
    x = g\brak{x}
\end{align}
For our problem,
\begin{align}
    g\brak{x} = \frac{2}{3}x^2 + \frac{5}{3}
\end{align}
Now the update equation will be,
\begin{align}
    x_{n+1} = g\brak{x_n} 
\end{align}
When we try to run the iterations however, we realize that whatever be the initial guess, 
the subsequent updated values grow without bound. This is becaue of the following theorem\\
\begin{align}
    \textbf{Theorem: }
\end{align}
    Let $x = s$ be a solution of $x = g\brak{x}$ and suppose that $g$ has a continuous
    derivative in some interval $J$ containing $s$. Then if $\abs{g^{\prime}} \le K < 1$ in $J$,
    the iteration process defined  above converges for any $x_0$ in $J$. The limit of the sequence
    $\sbrak{x_n}$ is s\\
\newline
Since there is no solution (evident by quadratic formula) there exists no interval J for which
the process converges to a point.\\
\newline
The same behaviour is shown by the Newton-Raphson Method,\\
Start with an initial guess $x_0$, and then run the following logical loop,
\begin{align}
    x_{n+1} = x_n - \frac{f\brak{x_n}}{f^{\prime}\brak{x_n}} 
\end{align}
The behaviour shown here is that regardless of which guess we take, it reaches a point of 
extrema(derivative $\approx$ 0) and then the process halts, or the updated point grow with bound.\\
To get the complex solutions, however , we can Just take the initial guess point to be a 
random complex number.\\
The ouput of a program written to find roots is shown below:
\begin{verbatim}
Running Newton iterations:
x got too big
Trying fixed point iterations:
x got too big
Trying complex Newton's iterations:
Solution = 0.750000 + -1.391941 i
\end{verbatim}
And on a second run, 
\begin{verbatim}
Running Newton iterations:
Failure
Trying fixed point iterations:
x got too big
Trying complex Newton's iterations:
Solution = 0.750000 + 1.391941 i
\end{verbatim}
\end{document}
