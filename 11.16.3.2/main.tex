\documentclass[journal]{IEEEtran}
\usepackage[a5paper, margin=10mm, onecolumn]{geometry}
%\usepackage{lmodern} % Ensure lmodern is loaded for pdflatex
\usepackage{tfrupee} % Include tfrupee package

\setlength{\headheight}{1cm} % Set the height of the header box
\setlength{\headsep}{0mm}     % Set the distance between the header box and the top of the text

\usepackage{gvv-book}
\usepackage{gvv}
\usepackage{cite}
\usepackage{amsmath,amssymb,amsfonts,amsthm}
\usepackage{algorithmic}
\usepackage{graphicx}
\usepackage{textcomp}
\usepackage{xcolor}
\usepackage{txfonts}
\usepackage{listings}
\usepackage{enumitem}
\usepackage{mathtools}
\usepackage{gensymb}
\usepackage{comment}
\usepackage[breaklinks=true]{hyperref}
\usepackage{tkz-euclide} 
\usepackage{listings}
% \usepackage{gvv}                                        
\def\inputGnumericTable{}                                 
\usepackage[latin1]{inputenc}                                
\usepackage{color}                                            
\usepackage{array}                                            
\usepackage{longtable}                                       
\usepackage{calc}                                             
\usepackage{multirow}                                         
\usepackage{hhline}                                           
\usepackage{ifthen}                                           
\usepackage{lscape}
\begin{document}

\bibliographystyle{IEEEtran}
\vspace{3cm}

\title{11.16.3.2}
\author{EE24BTECH11001 - Aditya Tripathy}
 \maketitle
% \newpage
% \bigskip
{\let\newpage\relax\maketitle}

\renewcommand{\thefigure}{\theenumi}
\renewcommand{\thetable}{\theenumi}
\setlength{\intextsep}{10pt} % Space between text and floats


\numberwithin{equation}{enumi}
\numberwithin{figure}{enumi}
\renewcommand{\thetable}{\theenumi}


\textbf{Question}:\\
A coin is tossed twice, what is the probability that atleast one tail occurs?
\\
\textbf{Solution: }\\
The sample space is 
\begin{align}
  \Omega = \cbrak{HH, HT, TH, TT}
\end{align}
Assuming equally likely outcomes, 
\begin{align}
  \Pr\brak{\omega \in \Omega} = \frac{1}{4}
\end{align}
Define a discrete random variable X = number of tails in the sequence.
\newline
Probability Mass Function $\Pr_{X}\brak{x}$ is given by:
\begin{align}
  \Pr_{X}\brak{x} = \begin{cases}
\frac{1}{4} & x = 0\\
    \frac{1}{2} & x = 1\\
    \frac{1}{4} & x = 2\\
\end{cases}
\end{align}
\begin{figure}[h!]
   \centering
   \includegraphics[width=0.7\columnwidth]{figs/fig1.png}
    \caption{Probability Mass Function}
\end{figure}
The CDF (Cumulative Distribution Function) is given by:
\begin{align}
  F_{X}\brak{x} = \Pr_X\brak{X \le x} = \begin{cases}
    0 & x < 0\\
    \frac{1}{4} & 0 \le x < 1\\
    \frac{3}{4} & 1 \le x < 2\\
    1 & x >=2
  \end{cases}
\end{align}
\begin{align}
  \Pr\brak{X \ge 1} &= 1 - \Pr\brak{X < 1}\\
  &= 1 - \frac{1}{4} = \frac{3}{4}
\end{align}
\begin{figure}[h!]
   \centering
   \includegraphics[width=0.7\columnwidth]{figs/fig2.png}
    \caption{Cumulative Distribution Function}
\end{figure}

Simulation:
\newline
To run a simulation we need to generate random numbers with uniform probability, which is done
as shown below(Algorithm taken from OpenSSL's random\_uniform.c):
\begin{enumerate}
  \item \text{Generate 32 bits of entropy using /dev/urandom.}
  \item Treat this as a fixed point number in the range [0, 1)
  \item Scale this to desired range using fixed point multiplication and treat as 64bit number(upper 32 bits integer and rest as fractional part)
  \item Return the integer part of the fixed point numbers
\end{enumerate}
The following shows how the relative frequency reaches true probability with increasing number of trials of the event.
\begin{figure}[h!]
   \centering
   \includegraphics[width=0.7\columnwidth]{figs/fig.png}
    \caption{Relative Frequency tends to True Probability}
\end{figure}
\newline
Approach 2 :
\newline
In this approach, we treat our random variable as the sum of outcomes of two bernoulli random
variables

\begin{align}
  X = X_1 + X_2
\end{align}
where,
\begin{align}
	X_i = \begin{cases}
		0 & \text{Outcome is Heads}\\	
		1 & \text{Outcome is Tails}	
	\end{cases}
\end{align}
\begin{align}
  \pr{X = x} = \begin{cases}
    p = \frac{1}{2} & x = 0\\
    1-p = \frac{1}{2} & x = 1
  \end{cases}
\end{align}
\begin{table}[htbp]
	\centering\begin{tabular}{|c|c|c|}
    \hline
		$X_1 \rightarrow X_2 \downarrow$ & 0 & 1\\
		\hline
		0 & $\frac{1}{4}$ & $\frac{1}{4}$\\
		\hline
		1 & $\frac{1}{4}$ & $\frac{1}{4}$\\
    \hline
	\end{tabular}
	\caption{Joint PMF Table}
\end{table}
More generally for $X_1 = a, X_2 = b$ associated probability is $p^a\brak{1-p}^b$
\newline
Now, 
\begin{align}
  \Pr\brak{X \ge 1} &= \Pr_{X_1 X_2}\brak{X_1 = 0, X_2 = 1} + \Pr_{X_1 X_2}\brak{X_1 = 1, X_2 = 0} + 
\Pr_{X_1 X_2}\brak{X_1 = 1, X_2 = 1}\\
  \Rightarrow \Pr\brak{X \ge 1} &= \frac{3}{4}\\
\end{align}
Before we tackle the general case we need to state the following theorm from combinatorics
\begin{align}
  x_1 + x_2 \cdots x_n = k, x_i \in \cbrak{0, 1}
\end{align}
has $\comb{n}{k}$ solutions.
\newline
In general if a Random variable is the sum of outcomes of $n$ Bernoulli random variables, then
\begin{align}
  \pr{X = k} = \comb{n}{k}p^{n-k}\brak{1-p}^{k} 
\end{align}
which is the famous binomial random variable.
\begin{figure}[h!]
   \centering
   \includegraphics[width=0.7\columnwidth]{figs/binomial.png}
    \caption{Generating binomial distribution from bernoulli}
\end{figure}
The blue plot is the $n = 2$ case and red plot is the $n = 10$ case.
\end{document}
